\documentclass{beamer}

\usepackage{colortbl}
\usepackage{tikz}
\usetikzlibrary{matrix,arrows,positioning,automata}

\definecolor{lgr}{rgb}{0.8,0.8,0.8}

%\newcommand{\cT}{{\cal T}}
\newcommand{\cS}{{\cal S}}

\newcommand{\compl}{\mathsf{c}}
\usepackage[ruled,vlined]{algorithm2e}


%------------------------------------------------------------------
% wrapping text around figures
%\usepackage{wrapfig}
%------------------------------------------------------------------

\newcommand{\B}[1]{\textbf{#1}}
\DeclareMathOperator*{\LW}{\bigg\rmoustache_{\cL}}
\newcommand{\cB}{{\cal B}}
\newcommand{\cA}{{\cal A}}
\newcommand{\cH}{{\cal H}}
\newcommand{\cN}{{\cal N}}
\newcommand{\cT}{{\cal T}}
\newcommand{\cC}{{\cal C}}
\newcommand{\sur}{\twoheadrightarrow}
\newcommand{\cL}{\mathcal{L}}
\newcommand{\gap}{\vskip10pt}

\setbeamertemplate{navigation symbols}{}


\usetheme{Boadilla}
\usecolortheme[rgb={.0,0.19,0.07}]{structure}
%\setbeamerfont{}{structuresmallcapsserif}
\useoutertheme{infolines}

\newcommand{\Magma}{\textsc{Magma}}
\newcommand{\GAP}{\textsc{GAP}}
\newcommand{\SgpDec}{\textsc{SgpDec}}
\newcommand{\Smallsemi}{\textsc{Smallsemi}}

\newcommand{\jump}{\vskip6pt}
\newcommand{\jmp}{\vskip3pt}


\begin{document}

\title[Subsemigroup Enumeration]{On Enumerating Subsemigroups of the Full Transformation Semigroup }
\author[e-n@]{Attila Egri-Nagy\\\ \\joint work with James East (UWS) and James D. Mitchell (University of St. Andrews, Scotland)}
\institute[UWS]{School of Computing, Engineering and Mathematics\\ University of Western Sydney}
\date[AUSTMS 56th]{2012.09.24. Australian Mathematical Society 56th Annual Meeting}

\begin{frame}
\titlepage
\end{frame}

\begin{frame}\frametitle{Motivation}
\begin{center}
For the holonomy decomposition of transformation semigroups we would like to know the maximum number of hierarchical levels for $n$ states.
\centerline{$\downarrow$}
We need examples of long decompositions to study.
\centerline{$\downarrow$}
Enumeration/classification of transformation semigroups.
\end{center}
\end{frame}

\begin{frame}\frametitle{Semigroups}
\begin{definition}
A \emph{semigroup} is a set with an associative binary operation.
\end{definition}

\begin{definition}
Transformation semigroup $(X,S)$ is a set of functions of a set $X$ to itself, closed under composition. 
\end{definition}
We will denote the transformation
$$
\left[ \begin{array}{ccccc}
1 & 2 & 3 & 4 & 5 \\
2 & 2 & 3 & 3 & 1
\end{array} \right]$$
simply by $[22331]$.
\end{frame}

\begin{frame}\frametitle{Semigroup enumeration and classification}

Problems:
\begin{itemize}
\item There are lots of semigroups.
\item Most of them are 3-nilpotent, i.e.\ they satisfy the $xyz=0$ identity.
\end{itemize}
\jump

\begin{quote}
``So, whereas groups are
gems, all of them precious, the garden of semigroups is filled with weeds. One
needs to yank out these weeds to find the interesting semigroups.''
\end{quote}
Rhodes, J., Steinberg, B.: The q-theory of Finite Semigroups. Springer (2008)
\end{frame}

\begin{frame}\frametitle{History of semigroup enumeration}
\begin{enumerate}
\item[1955] Forsythe, G. E., \textbf{\emph{SWAC computes 126 distinct semigroups of order 4}}, Proc. Amer. Math. Soc., 6 (1955), 443--447.
\\\vskip6pt
Tetsuya, K., Hashimoto, T., Akazawa, T., Shibata, R., Inui, T. and Tamura, T., \textbf{{All semigroups of order at most 5}}, \emph{J. Gakugei Tokushima Univ. Nat. Sci. Math.}, 6 (1955), 19--39.
\item[1967] Plemmons, R. J., \textbf{There are 15973 semigroups of order 6}, \emph{Math. Algorithms}, 2 (1967), 2--17. 
\item[1977]J\"urgensen, H. and Wick, P., \textbf{Die Halbgruppen der Ordnungen  $\mathbf{\leq}$ 7}, \emph{Semigroup Forum}, 14 (1) (1977), 69--79.
\item[1994]Satoh, S., Yama, K. and Tokizawa, M., \textbf{Semigroups of order 8}, \emph{Semigroup Forum}, 49 (1) (1994), 7--29.  
\end{enumerate}
\end{frame}

\begin{frame}\frametitle{Current state of semigroup enumeration}
Inspired by the  \textsc{SmallGroups Library}\ for \GAP\ and \Magma\ there is now a \GAP\ package called \Smallsemi.
\jump
\Smallsemi\ provides a database of all the small semigroups up to order 8, tools for identifying semigroups and their properties (e.g.\ commutative, band, inverse, regular, etc., 16 of them in total ).  
\jump
The size of the compressed database is 22 Mbytes.
\jump
Andreas Distler, James D. Mitchell
\jmp
\url{http://www-groups.mcs.st-andrews.ac.uk/~jamesm/smallsemi/}
\end{frame}


\begin{frame}\frametitle{Number of semigroups of order $n$}
\begin{center}
\begin{tabular}{r|r|r|r}
order & \#groups &\#semigroups &\#3-nilpotent semigroups \\
\hline
1&1&1 &0\\
2&1&4 &0\\
3&1&18&1\\
4&2&126&8\\
5&1&1,160&84\\
6&2&15,973&2,660\\
7&1&836,021&609,797\\
8&5&1,843,120,128&1,831,687,022\\
9&2&52,989,400,714,478&52,966,239,062,973
\end{tabular}
\end{center}
The calculation was done by combining \GAP\ and a Constraint Satisfaction Problem (CSP) solver Minion \url{minion.sf.net}.
\end{frame}

\begin{frame}\frametitle{Enumerating transformation semigroups}
Idea: Find the subsemigroups of the full transformation semigroup.
\jump
Straightforward brute-force algorithm: enumerate all subsets of $\cT_n$ and keep those that form a subsemigroup.
\jump
However, there are $2^{n^n}$ subsets of $\cT_n$.
\begin{center}
\renewcommand{\arraystretch}{1.5}

\begin{tabular}{l|l|l}
$n$&$n^n$&$2^{n^n}$\\
\hline
1 & 1 & 2 \\
\hline
2 & 4 & 16 \\
\hline
3 & 27 & 134217728 \\
\hline\hline\hline
4 & 256 & \parbox[l]{.5\textwidth}{11579208923731619542357098500\\86879078532699846656405640394\\57584007913129639936}\\
\hline
5 & 3125 & $2^{3125}$
\end{tabular}
\end{center}
\end{frame}

\begin{frame}\frametitle{Idea: systematic reduction of multiplication tables}
Let $(X,S)$ be a transformation semigroup, $n=|S|$. We fix an order on the semigroup elements, $s_1,\ldots, s_n$, thus we can easily refer to the elements by their indices. 
\begin{definition}
Then the  \emph{multiplication table} of $S$ is a $n\times n$ matrix $M$ with entries from $\{1,..,n\}$ such that $M_{i,j}=k$ if $s_is_j=s_k$. This table is often called the \emph{Cayley-table} of the semigroup.
\end{definition}

\begin{definition}[cut, closed cut]
A \emph{cut} is a subset of the semigroup, $K\subseteq S$ a set elements that we cut from the $M$.  A cut is \emph{closed} if the table spanned by $S\setminus K$ is a multiplication table, i.e.\ it is closed under multiplication.
\end{definition}
%Why this approach? 
%\jump
%To keep it general and save time (well, in some sense).
\end{frame}

\begin{frame}\frametitle{Completions of a cut}

\begin{definition}[\textbf{completion of a cut}]
$$C(K)=\{i\in S\setminus K \mid\ \exists j\in S\setminus K \text{ such that } M_{i,j}\in K \text{ or } M_{j,i}\in K\} $$
\noindent i.e.\ those elements not in the cut, whose column or row contains an element in the cut.
\end{definition}

\begin{definition}[\textbf{diagonal completion of a cut}]
$$D(K)=\{i\in S\setminus K \mid\ M_{i,i}\in K \} $$
\noindent i.e.\ those elements not in the cut, whose diagonal contains an element in the cut.
\end{definition}
\end{frame}

\begin{frame}\frametitle{Example: $\cS_3$}
Consider $\cS_3$ with the ordering:  (), (2,3), (1,2), (1,2,3), (1,3,2), (1,3).

The cut $K=\{2\}$ (i.e.\ removing (2,3)) is not a closed one. 

$C(K)=\{3,4,5,6\}$

$K$ extended by the completion  $K\cup C(K)=\{2,3,4,5,6\}$ happens to be closed.

\begin{center}
\setlength{\fboxsep}{1pt}
\begin{tabular}{cccccc}
1&\color{lgr}2&3&4&5&6\\
\color{lgr}2&\color{lgr}1&\color{lgr}4&\color{lgr}3&\color{lgr}6&\color{lgr}5\\
3&\color{lgr}5&1&6&\color{white}\colorbox{black}{2}&4\\
4&\color{lgr}6&\color{white}\colorbox{black}{2}&5&1&3\\
5&\color{lgr}3&6&1&4&\color{white}\colorbox{black}{2}\\
6&\color{lgr}4&5&\color{white}\colorbox{black}{2}&3&1\\
\end{tabular}
\end{center}
\end{frame}

\begin{frame}\frametitle{Problem}
The closed cut $K\cup C(K)$ corresponds to the trivial subgroup.  However there are more closed cuts including $K$: $\{2,3,6\}$, $\{2,3,4,5\}$,$\{2,4,5,6\}$.
\begin{center}
\begin{tabular}{@{}c@{}c@{}c@{}c@{}c@{}c@{}}
1&\color{lgr}2&\color{lgr}3&4&5&\color{lgr}6\\
\color{lgr}2&\color{lgr}1&\color{lgr}4&\color{lgr}3&\color{lgr}6&\color{lgr}5\\
\color{lgr}3&\color{lgr}5&\color{lgr}1&\color{lgr}6&\color{lgr}2&\color{lgr}4\\
4&\color{lgr}6&\color{lgr}2&5&1&\color{lgr}3\\
5&\color{lgr}3&\color{lgr}6&1&4&\color{lgr}2\\
\color{lgr}6&\color{lgr}4&\color{lgr}5&\color{lgr}2&\color{lgr}3&\color{lgr}1\\
\end{tabular}\ \ \ \ 
\begin{tabular}{@{}c@{}c@{}c@{}c@{}c@{}c@{}}
1&\color{lgr}2&\color{lgr}3&\color{lgr}4&\color{lgr}5&6\\
\color{lgr}2&\color{lgr}1&\color{lgr}4&\color{lgr}3&\color{lgr}6&\color{lgr}5\\
\color{lgr}3&\color{lgr}5&\color{lgr}1&\color{lgr}6&\color{lgr}2&\color{lgr}4\\
\color{lgr}4&\color{lgr}6&\color{lgr}2&\color{lgr}5&\color{lgr}1&\color{lgr}3\\
\color{lgr}5&\color{lgr}3&\color{lgr}6&\color{lgr}1&\color{lgr}4&\color{lgr}2\\
6&\color{lgr}4&\color{lgr}5&\color{lgr}2&\color{lgr}3&1\\
\end{tabular}\ \ \ \ 
\begin{tabular}{@{}c@{}c@{}c@{}c@{}c@{}c@{}}
1&\color{lgr}2&3&\color{lgr}4&\color{lgr}5&\color{lgr}6\\
\color{lgr}2&\color{lgr}1&\color{lgr}4&\color{lgr}3&\color{lgr}6&\color{lgr}5\\
3&\color{lgr}5&1&\color{lgr}6&\color{lgr}2&\color{lgr}4\\
\color{lgr}4&\color{lgr}6&\color{lgr}2&\color{lgr}5&\color{lgr}1&\color{lgr}3\\
\color{lgr}5&\color{lgr}3&\color{lgr}6&\color{lgr}1&\color{lgr}4&\color{lgr}2\\
\color{lgr}6&\color{lgr}4&\color{lgr}5&\color{lgr}2&\color{lgr}3&\color{lgr}1\\
\end{tabular}%\ \ \ \ 
%\begin{tabular}{@{}c@{}c@{}c@{}c@{}c@{}c@{}}
%1&\color{lgr}2&\color{lgr}3&4&5&\color{lgr}6\\
%\color{lgr}2&\color{lgr}1&\color{lgr}4&\color{lgr}3&\color{lgr}6&\color{lgr}5\\
%\color{lgr}3&\color{lgr}5&\color{lgr}1&\color{lgr}6&\color{lgr}2&\color{lgr}4\\
%4&\color{lgr}6&\color{lgr}2&5&1&\color{lgr}3\\
%5&\color{lgr}3&\color{lgr}6&1&4&\color{lgr}2\\
%\color{lgr}6&\color{lgr}4&\color{lgr}5&\color{lgr}2&\color{lgr}3&\color{lgr}1\\
%\end{tabular}
\end{center} 

This means that we have to extend the cut one by one with the elements from the completion. Therefore we are back to the brute-force algorithm (actually even less efficient).
\end{frame}

\begin{frame}\frametitle{The diagonal closure of a cut}
Iterating 
$$ \Delta(K):=K\cup D(K)$$
Since cutting an element from a diagonal can be done only one way, we can extend the cut by its diagonal completion.

Again using the multiplication table of $\cS_3$ if we cut by $K=\{5\}$ we get the following table:
\begin{center}
\begin{tabular}{@{}c@{}c@{}c@{}c@{}c@{}c@{}}
1&2&3&4&\color{lgr}5&6\\
2&1&4&3&\color{lgr}6&\color{white}\colorbox{black}{5}\\
3&\color{white}\colorbox{black}{5}&1&6&\color{lgr}2&4\\
4&6&2&\color{white}\colorbox{black}{5}&\color{lgr}1&3\\
\color{lgr}5&\color{lgr}3&\color{lgr}6&\color{lgr}1&\color{lgr}4&\color{lgr}2\\
6&4&\color{white}\colorbox{black}{5}&2&\color{lgr}3&1\\
\end{tabular}
\end{center}
5 appears in the diagonal for element 4, so $\Delta(\{5\})=\{4,5\}$. In this particular case $\Delta(\{4\})$ is also $\{4,5\}$, but having the same closure is not a symmetric relation. For instance, $\Delta(\{1\})=\{1,2,3,6\}$ but $\Delta(\{6\})=\{6\}$. 

\end{frame}

\begin{frame}[fragile]
\begin{algorithm}[H]
\SetKwInOut{Input}{input}\SetKwInOut{Output}{output}
\SetKwData{finished}{finished}
\SetKw{true}{true}
\SetKw{false}{false}
\Input{$M$ multiplication table, $K$ a cut}
\Output{$K$ extended to $\Delta(K)$}
\Repeat{\finished}{
  \finished $\leftarrow$ \true\;
  \For{$i\in S\setminus K$}{
    \If{$M_{i,i}\in K$}{
      $K\leftarrow K\cup \{i\}$\;
      \finished $\leftarrow$ \false\;
    }
  }
}
\caption{Calculating the diagonal closure of a cut.}
\label{alg:diagonalclosure}
\end{algorithm}
\jump
Is it possible to have overlapping diagonal closures?
\end{frame}

\begin{frame}[fragile]
\begin{algorithm}[H]
\SetKwInOut{Input}{input}\SetKwInOut{Output}{output}
\SetKwData{Subs}{subs}
\SetKwData{Visited}{visited}
\SetKwFunction{Reduce}{Reduce}
\Input{$M$ multiplication table, $K$ a cut}
\Output{closed cuts added to the collection \Subs}
\BlankLine

\Visited $\leftarrow$ \Visited$\cup \{K\}$\;
\If{$C(K)=\varnothing$}{\Subs$\leftarrow$\Subs$\cup\{K\}$\;}
\For{$i\in C(K)$}{
    $K\leftarrow$ $\Delta(K\cup\{i\})$\;
    \If{$K\notin$ \Visited}{
      \Reduce($M,K$)\;
    }
}
\caption{\texttt{Reduce}($M,K$), the recursive reduction algorithm.  Diagonal closure applied.}
\label{alg:basicrecursive}
\end{algorithm}
\pause
Still, this is not enough to attack $\cT_4$... 
\end{frame}


\begin{frame}\frametitle{Exploiting symmetries}
We use the most traditional approach to conjugacy for semigroups  and define \emph{G-conjugacy}. Elements $s,t\in S$ are $G$-conjugate, denoted by
$$s\sim_G t, \text{ if } s=g^{-1}tg \text{ for some } g\in G.$$ 
\jump
Here we act on the transformation representation.
For full transformation semigroups we can use the symmetric groups, otherwise group components from the holonomy decomposition can be used (it is capable of detecting the symmetries of semigroups of order up to $\approx$100,000). 
\jump
Ways to use conjugacy:
\begin{itemize}
\item Whenever we find a subsemigroup we take the orbit under conjugation.
\item For a non-semigroup subset we can also use the conjugacy class to prune the underlying search tree. 
\item We start cutting only from conjugacy class representatives.
\end{itemize}
\ldots and of course we get the conjugacy classes as well.
\end{frame}

\begin{frame}\frametitle{``Rescuing elements''}
Observation: There is a problem with trying to cut the identity from groups. After the diagonal closure the algorithm reverts back to full enumeration of the subsets of $S\setminus \Delta(K)$.
\jump
Instead of cutting with $K\cup\{i\}$, $i\in C(K)$, we try to find $C'(K)\subseteq C(K)\setminus \{i\}$ such that $K\cup C'(K)$ is closed.

\jump
Without rescuing searching in $\cT_3$ visits 1505328 subsets (12.9MB memory) and has 2629323 revisits.

With rescuing 15664 visits (138.33KB) and 19143 duplicates.  
\jump
So far this version of the algorithm works best...
\end{frame}

\begin{frame}\frametitle{How to measure complexity/efficiency?}
The number of visited cuts - the space complexity.
\jump
The number of visited cuts and the number of revisits.
\end{frame}


\begin{frame}\frametitle{Easy test cases: Cyclic Groups}

Cyclic groups - the number of subgroups is the number of divisors.
\jump
Cyclic groups of prime order - just 2 subgroups, but there is a bit of surprise.
\begin{tabular}{c|c|c|c|c|c|c|c|c|c|c|c|c|}
$n$ & 2 & 3 & 5 & 7 & 11 & 13 & 17 & 19 & 23 & 29 & 31 & 37  \\
\hline
\#visited & 2 & 3&  3& 7& 3& 3& 7& 3& 7& 3& 127 & 3 
\end{tabular}
\jump
\begin{tabular}{c|c|c|c|c|c|c|c|c|c|c|c|c|}
$n$ & 41 & 43 & 47 & 53 & 59 & 61 & 67 & 71 & 73 & 79 & 83 & 89  \\
\hline
\#visited & 7 & 15 &  7& 3& 3& 3& 3& 7& 511& 7& 3 & 511 
\end{tabular}
\jump
\begin{tabular}{c|c|c|c|c|c|c|c|c|c|}
$n$ & 97 & 101 & 103 & 107 & 109 & 113 & 127 & 131 & 137   \\
\hline
\#visited & 7 & 3 &  7& 3& 15 & 31 & 524287 & 3& 7
\end{tabular}


\end{frame}




\begin{frame}\frametitle{All subsemigroups of $\cT_2$}
$1\mapsto[1,1]$, $2\mapsto[1,2]$, $3\mapsto[2,1]$, $4\mapsto[2,2]$ 
\renewcommand{\arraystretch}{0.3}
\begin{center}
%{}
\begin{tabular}{@{}c@{}c@{}c@{}c@{}}
1&1&4&4\\
1&2&3&4\\
1&3&2&4\\
1&4&1&4\\
\end{tabular},\ \ \ 
%{3}
\begin{tabular}{@{}c@{}c@{}c@{}c@{}}
1&1&\color{lgr}4&4\\
1&2&\color{lgr}3&4\\
\color{lgr}1&\color{lgr}3&\color{lgr}2&\color{lgr}4\\
1&4&\color{lgr}1&4\\
\end{tabular},\ \ \ 
%{3,1}
\begin{tabular}{@{}c@{}c@{}c@{}c@{}}
\color{lgr}1&\color{lgr}1&\color{lgr}4&\color{lgr}4\\
\color{lgr}1&2&\color{lgr}3&4\\
\color{lgr}1&\color{lgr}3&\color{lgr}2&\color{lgr}4\\
\color{lgr}1&4&\color{lgr}1&4\\
\end{tabular},\ \ \ 
%{3,1,2}
\begin{tabular}{@{}c@{}c@{}c@{}c@{}}
\color{lgr}1&\color{lgr}1&\color{lgr}4&\color{lgr}4\\
\color{lgr}1&\color{lgr}2&\color{lgr}3&\color{lgr}4\\
\color{lgr}1&\color{lgr}3&\color{lgr}2&\color{lgr}4\\
\color{lgr}1&\color{lgr}4&\color{lgr}1&4\\
\end{tabular},\ \ \ 
%{3,4}
\begin{tabular}{@{}c@{}c@{}c@{}c@{}}
1&1&\color{lgr}4&\color{lgr}4\\
1&2&\color{lgr}3&\color{lgr}4\\
\color{lgr}1&\color{lgr}3&\color{lgr}2&\color{lgr}4\\
\color{lgr}1&\color{lgr}4&\color{lgr}1&\color{lgr}4\\
\end{tabular},\ \ \ 
%{2,3}
\begin{tabular}{@{}c@{}c@{}c@{}c@{}}
1&\color{lgr}1&\color{lgr}4&4\\
\color{lgr}1&\color{lgr}2&\color{lgr}3&\color{lgr}4\\
\color{lgr}1&\color{lgr}3&\color{lgr}2&\color{lgr}4\\
1&\color{lgr}4&\color{lgr}1&4\\
\end{tabular},\ \ \ 
%{3,4,2}
\begin{tabular}{@{}c@{}c@{}c@{}c@{}}
1&\color{lgr}1&\color{lgr}4&\color{lgr}4\\
\color{lgr}1&\color{lgr}2&\color{lgr}3&\color{lgr}4\\
\color{lgr}1&\color{lgr}3&\color{lgr}2&\color{lgr}4\\
\color{lgr}1&\color{lgr}4&\color{lgr}1&\color{lgr}4\\
\end{tabular},\ \ \ 
%{1,4}
\begin{tabular}{@{}c@{}c@{}c@{}c@{}}
\color{lgr}1&\color{lgr}1&\color{lgr}4&\color{lgr}4\\
\color{lgr}1&2&3&\color{lgr}4\\
\color{lgr}1&3&2&\color{lgr}4\\
\color{lgr}1&\color{lgr}4&\color{lgr}1&\color{lgr}4\\
\end{tabular},\ \ \ 
%{3,1,4}
\begin{tabular}{@{}c@{}c@{}c@{}c@{}}
\color{lgr}1&\color{lgr}1&\color{lgr}4&\color{lgr}4\\
\color{lgr}1&2&\color{lgr}3&\color{lgr}4\\
\color{lgr}1&\color{lgr}3&\color{lgr}2&\color{lgr}4\\
\color{lgr}1&\color{lgr}4&\color{lgr}1&\color{lgr}4\\
\end{tabular}
\end{center}

%\end{frame}

%\begin{frame}
\begin{tikzpicture}[scale=.6,>=latex',line join=bevel,]

  \pgfsetlinewidth{1bp}
%%
\pgfsetcolor{black}
  % Edge: k13 -> k123
  \draw [->] (76.664bp,71.831bp) .. controls (84.973bp,63.285bp) and (95.026bp,52.944bp)  .. (111.1bp,36.413bp);
  % Edge: k23 -> k234
  \draw [->] (211.4bp,71.831bp) .. controls (219.04bp,63.369bp) and (228.27bp,53.149bp)  .. (243.38bp,36.413bp);
  % Edge: k3 -> k23
  \draw [->] (288.46bp,143.83bp) .. controls (272.32bp,134.54bp) and (252.5bp,123.12bp)  .. (226.53bp,108.16bp);
  % Edge: k -> k3
  \draw [->] (373.08bp,215.83bp) .. controls (364.66bp,207.28bp) and (354.46bp,196.94bp)  .. (338.16bp,180.41bp);
  % Edge: k34 -> k234
  \draw [->] (313.08bp,71.831bp) .. controls (304.66bp,63.285bp) and (294.46bp,52.944bp)  .. (278.16bp,36.413bp);
  % Edge: k14 -> k134
  \draw [->] (462bp,143.83bp) .. controls (462bp,136.13bp) and (462bp,126.97bp)  .. (462bp,108.41bp);
  % Edge: k3 -> k34
  \draw [->] (322.78bp,143.83bp) .. controls (323.95bp,136.13bp) and (325.35bp,126.97bp)  .. (328.19bp,108.41bp);
  % Edge: k -> k14
  \draw [->] (408.92bp,215.83bp) .. controls (417.34bp,207.28bp) and (427.54bp,196.94bp)  .. (443.84bp,180.41bp);
  % Edge: k3 -> k13
  \draw [->] (254.81bp,144.02bp) .. controls (216.12bp,133.34bp) and (167.1bp,119.82bp)  .. (118.01bp,106.28bp);
  % Edge: k23 -> k123
  \draw [->] (178.35bp,71.831bp) .. controls (170.59bp,63.369bp) and (161.22bp,53.149bp)  .. (145.88bp,36.413bp);
  % Edge: k3 -> k134
  \draw [->] (355.83bp,143.83bp) .. controls (374.42bp,134.41bp) and (397.3bp,122.81bp)  .. (426.18bp,108.16bp);
  % Node: k13
\begin{scope}
  \definecolor{strokecol}{rgb}{0.0,0.0,0.0};
  \pgfsetstrokecolor{strokecol}
  \draw (59bp,90bp) node {\scriptsize$\{1,3\}\mapsto\{[12],[22]\}$};
\end{scope}
  % Node: k34
\begin{scope}
  \definecolor{strokecol}{rgb}{0.0,0.0,0.0};
  \pgfsetstrokecolor{strokecol}
  \draw (331bp,90bp) node {\scriptsize$\{3,4\}\mapsto\{[11],[12]\}$};
\end{scope}
  % Node: k14
\begin{scope}
  \definecolor{strokecol}{rgb}{0.0,0.0,0.0};
  \pgfsetstrokecolor{strokecol}
  \draw (462bp,162bp) node {\scriptsize$\{1,4\}\mapsto\{[12],[21]\}$};
\end{scope}
  % Node: k23
\begin{scope}
  \definecolor{strokecol}{rgb}{0.0,0.0,0.0};
  \pgfsetstrokecolor{strokecol}
  \draw (195bp,90bp) node {\scriptsize$\{2,3\}\mapsto\{[11],[22]\}$};
\end{scope}
  % Node: k
\begin{scope}
  \definecolor{strokecol}{rgb}{0.0,0.0,0.0};
  \pgfsetstrokecolor{strokecol}
  \draw (391bp,234bp) node {\scriptsize$\{\}\mapsto\{[11],[12],[21],[22]\}$};
\end{scope}
  % Node: k234
\begin{scope}
  \definecolor{strokecol}{rgb}{0.0,0.0,0.0};
  \pgfsetstrokecolor{strokecol}
  \draw (260bp,18bp) node {\scriptsize$\{2,3,4\}$, \{[11]\}};
\end{scope}
  % Node: k3
\begin{scope}
  \definecolor{strokecol}{rgb}{0.0,0.0,0.0};
  \pgfsetstrokecolor{strokecol}
  \draw (320bp,162bp) node {\scriptsize$\{3\}\mapsto\{[11],[12],[22]\}$};
\end{scope}
  % Node: k123
\begin{scope}
  \definecolor{strokecol}{rgb}{0.0,0.0,0.0};
  \pgfsetstrokecolor{strokecol}
  \draw (129bp,18bp) node {\scriptsize$\{1,2,3\}\mapsto\{[22]\}$};
\end{scope}
  % Node: k134
\begin{scope}
  \definecolor{strokecol}{rgb}{0.0,0.0,0.0};
  \pgfsetstrokecolor{strokecol}
  \draw (462bp,90bp) node {\scriptsize$\{1,3,4\}\mapsto\{[12]\}$};
\end{scope}
%
\end{tikzpicture}

\end{frame}
\begin{frame}\frametitle{All subsemigroups of $\cT_2$}

\renewcommand{\arraystretch}{0.3}
\begin{center}
%{}
\begin{tabular}{@{}c@{}c@{}c@{}c@{}}
1&1&4&4\\
1&2&3&4\\
1&3&2&4\\
1&4&1&4\\
\end{tabular}\\\jmp 
%{3}
\begin{tabular}{@{}c@{}c@{}c@{}c@{}}
1&1&\color{lgr}4&4\\
1&2&\color{lgr}3&4\\
\color{lgr}1&\color{lgr}3&\color{lgr}2&\color{lgr}4\\
1&4&\color{lgr}1&4\\
\end{tabular}\\\jmp 
%{1,4}
\begin{tabular}{@{}c@{}c@{}c@{}c@{}}
\color{lgr}1&\color{lgr}1&\color{lgr}4&\color{lgr}4\\
\color{lgr}1&2&3&\color{lgr}4\\
\color{lgr}1&3&2&\color{lgr}4\\
\color{lgr}1&\color{lgr}4&\color{lgr}1&\color{lgr}4\\
\end{tabular}\\\jmp

%{3,1}
\begin{tabular}{@{}c@{}c@{}c@{}c@{}}
\color{lgr}1&\color{lgr}1&\color{lgr}4&\color{lgr}4\\
\color{lgr}1&2&\color{lgr}3&4\\
\color{lgr}1&\color{lgr}3&\color{lgr}2&\color{lgr}4\\
\color{lgr}1&4&\color{lgr}1&4\\
\end{tabular},\ \ \ 
%{3,4}
\begin{tabular}{@{}c@{}c@{}c@{}c@{}}
1&1&\color{lgr}4&\color{lgr}4\\
1&2&\color{lgr}3&\color{lgr}4\\
\color{lgr}1&\color{lgr}3&\color{lgr}2&\color{lgr}4\\
\color{lgr}1&\color{lgr}4&\color{lgr}1&\color{lgr}4\\
\end{tabular}\\\jmp

%{2,3}
\begin{tabular}{@{}c@{}c@{}c@{}c@{}}
1&\color{lgr}1&\color{lgr}4&4\\
\color{lgr}1&\color{lgr}2&\color{lgr}3&\color{lgr}4\\
\color{lgr}1&\color{lgr}3&\color{lgr}2&\color{lgr}4\\
1&\color{lgr}4&\color{lgr}1&4\\
\end{tabular}\\\jmp 
%{3,1,4}
\begin{tabular}{@{}c@{}c@{}c@{}c@{}}
\color{lgr}1&\color{lgr}1&\color{lgr}4&\color{lgr}4\\
\color{lgr}1&2&\color{lgr}3&\color{lgr}4\\
\color{lgr}1&\color{lgr}3&\color{lgr}2&\color{lgr}4\\
\color{lgr}1&\color{lgr}4&\color{lgr}1&\color{lgr}4\\
\end{tabular}\ \ \ \ \ \ \ \ \ \ \ \ \ 
%{3,4,2}
\begin{tabular}{@{}c@{}c@{}c@{}c@{}}
1&\color{lgr}1&\color{lgr}4&\color{lgr}4\\
\color{lgr}1&\color{lgr}2&\color{lgr}3&\color{lgr}4\\
\color{lgr}1&\color{lgr}3&\color{lgr}2&\color{lgr}4\\
\color{lgr}1&\color{lgr}4&\color{lgr}1&\color{lgr}4\\
\end{tabular},\ \ \ 
%{3,1,2}
\begin{tabular}{@{}c@{}c@{}c@{}c@{}}
\color{lgr}1&\color{lgr}1&\color{lgr}4&\color{lgr}4\\
\color{lgr}1&\color{lgr}2&\color{lgr}3&\color{lgr}4\\
\color{lgr}1&\color{lgr}3&\color{lgr}2&\color{lgr}4\\
\color{lgr}1&\color{lgr}4&\color{lgr}1&4\\
\end{tabular},\ \ \ 


\end{center}
\end{frame}



\begin{frame}\frametitle{Summary of Results}
\begin{center}

\begin{tabular}{|r|c|c|c|c|c|}
\hline
$n$ & 0 & 1 & 2 & 3 & 4 \\
\hline
$S_n$ & - & 1,1 & 2,2 & 6,4 & 30,11\\
\hline
$T_n$ &1,1 & 2,2 & 10,8 & 1299,283 & \\
\hline
$T_n\setminus S_n$ & 1,1 & 1,1 & 4,3 & 600,123& \\
\hline
\end{tabular}
\end{center}
A215650, A215651 \url{http://oeis.org}
\end{frame}

\begin{frame}\frametitle{What can we do?}

\begin{itemize}
\item Wait.
\item Have a closer look at what the algorithm is actually doing.
\item We know what are the maximal subsemigroups of $\cT_n$: either a maximal subgroup of $\cS_n$ and every other non-invertible function, or the whole $\cS_n$ and all the functions with at most $n-2$ points in the image. Do the reduction starting from those then merge the partial results.
\end{itemize}
\end{frame}


\begin{frame}\frametitle{What to expect?}
A database of all transformation semigroups on $n$ points.
\begin{itemize}
\item $n\leq 3$ we have the data, included in the \textsc{GAP} package \textsc{Semigroups}.
\item $n=4$ It seems to be within reach with the same heuristics, just a bit more data juggling.
\item $n=5$ Probably the same idea may work with more new heuristics and solving big data handling difficulties.
\item $n=6$ Not with this idea.
\end{itemize}
\end{frame}


\begin{frame}
\begin{center}\Huge Thank You!\end{center}
\normalsize

Group \& semigroup decomposition software:
\begin{center}
\url{http://sgpdec.sf.net}
\end{center}

On computational semigroup theory:
\begin{center}
\url{http://compsemi.wordpress.com}
\end{center}


\end{frame}


\end{document}

