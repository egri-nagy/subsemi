%%%%%%%%%%%%%%%%%%%%%%%%%%%%%%%%%%%%%%%%%%%%%%%%%%%%%%%%%%%%%%
% AAA83 abstract template file  aaa83_template.tex

\documentclass[11pt,reqno]{amsart}

%%%%%%%%%%%%%%%%%%%%%%%%%%%%%%%%%%%%%%%%%%%%%%%%%%%%%%%%%%%%%%
% These packages will be used in the composition
% of the booklet of abstracts. If your LaTeX system
% does not have the `mathpazo' font, you can easily
% install it (if it is not done by your system
% automatically), as it is a standard part of most
% of the LaTeX distributions. If you have any problems,
% however, feel free to comment out the `\usepackage{mathpazo}'
% line and write your abstract by using the default font
% -- we will complete the text processing anyhow.

\usepackage{amssymb}
\usepackage{amsmath}
\usepackage{mathpazo}
\usepackage{calc}

%%%%%%%%%%%%%%%%%%%%%%%%%%%%%%%%%%%%%%%%%%%%%%%%%%%%%%%%%%%%%%
% If you want to include any further packages, list them here.



%%%%%%%%%%%%%%%%%%%%%%%%%%%%%%%%%%%%%%%%%%%%%%%%%%%%%%%%%%
% Page formatting commands, producing appropriate margins.
% Please, do not change them.

\setlength{\paperheight}{29.7cm} \setlength{\paperwidth}{21cm} \setlength{\textheight}{18.5cm}
\setlength{\textwidth}{12.2cm} \setlength{\headheight}{14.1pt} \setlength{\headsep}{19pt}
\setlength{\topmargin}{5.6cm-16.55pt-1in} \setlength{\oddsidemargin}{4.4cm-1in} \setlength{\evensidemargin}{4.4cm-1in}
\addtolength{\footskip}{-4pt}

\pagestyle{empty}

%%%%%%%%%%%%%%%%%%%%%%%%%%%%%%%%%%%%%%%%%%%%%%%%%%%%%%%%%%%%%%
% The body of the LaTeX document starts here.

\begin{document}

\setlength{\parindent}{0pt} \thispagestyle{empty}

% TITLE

\begin{flushleft}

\large

\textbf{On Enumerating Transformation Semigroups}

\end{flushleft}

% THE HEADLINE

\normalsize

\vspace{-2.25mm}

\rule{\textwidth}{1.5pt}

\vspace{2mm}

% THE AUTHOR

\begin{flushright}

\large

\textsc{Attila Egri-Nagy}

\normalsize

\vspace{3pt}

% AFFILIATION - The format is: unit, institution, CITY (country code will be added by us)

\scriptsize

\textit{Centre for Research in Mathematics}

\textit{University of Western Sydney, SYDNEY}

\texttt{a.egri-nagy@uws.edu.au}

\end{flushright}

\normalsize

\setlength{\parskip}{11pt}\baselineskip=1.05\baselineskip

% THE ABSTRACT

Motivated by other algorithmic problems, we are aiming to enumerate transformation semigroups on $n$ points by finding all subsemigroups of the full transformation semigroup $T_n$.

Pen and paper calculation shows that there are $10$ subsemigroups of $T_2$ in $8$ conjugacy classes.
Brute force computer calculation (checking all subsets) gives the answer for $n=3$: there are 1299 subsemigroups in 283 conjugacy classes.

Due to the huge search space, for $n=4$,  we have to use a different method.
With some heuristics applied we recursively reduce the multiplication table of the semigroup.
Computations are under way, but we know that there are (at least) 3788252 subsemigroups in 162332 conjugacy classes of $K_{4,2}$, the semigroup of all transformations on 4 points with image size of maximum 2.
Due to the generality of the method, we can enumerate the subsemigroups of an arbitrary transformation semigroup and we can ask special questions like ``What are the semigroups that contain no constant maps?''. This can be used for enumerating automata with no synchronizing words.

In this talk we describe the reduction algorithm and the obtained data sets in more detail.
This is a joint work with {\sc James D. Mitchell} (University of St Andrews) and {\sc James East} (University of Western Sydney).

\end{document}
