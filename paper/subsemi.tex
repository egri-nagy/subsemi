\documentclass{amsart}


\newcommand{\cT}{{\mathcal T}}
\newcommand{\cS}{{\mathcal S}}
\newcommand{\Sub}{\mathbf{Sub}}
\newcommand{\Max}{\mathbf{Max}}
\newcommand{\compl}{\mathsf{c}}

\DeclareMathOperator{\Aut}{Aut}

\newcommand{\todo}[1]{ \small \textsf{[TODO:  #1 ]} \normalsize}


\theoremstyle{plain}
\newtheorem{theorem}{Theorem}[section]
\newtheorem{lemma}[theorem]{Lemma}
\newtheorem{fact}[theorem]{Fact}
\newtheorem{Proposition}[theorem]{Proposition}
\newtheorem{cor}[theorem]{Corollary}
\theoremstyle{definition}
\newtheorem{definition}[theorem]{Definition}
\newtheorem{example}[theorem]{Example}

\newcommand{\SgpDec}{\textsc{SgpDec}}
\newcommand{\GAP}{\textsc{Gap}}
\newcommand{\Viz}{\textsc{Viz}}
\newcommand{\GraphViz}{\textsc{GraphViz}}



\begin{document}
\section{Notation}
%Let $(X,S)$ be a transformation semigroup, $n=|S|$.
Let $S$ be a semigroup, $n=|S|$.
We fix an order on the semigroup elements, $s_1,\ldots, s_n$, so we can easily refer to the elements by their indices. 
Then the  \emph{multiplication table}, or \emph{Cayley-table} of $S$ is a $n\times n$ matrix $M$ with entries from $\{1,..,n\}$ such that $M_{i,j}=k$ if $s_is_j=s_k$.
$\Sub(S)=\big\{T\mid T\leq S \big\}$ the set of all subsemigroups, $\Max(S)$ the set of maximal proper subsemigroups.

\section{Useful facts}
We can simplify and parallelize enumerating subsemigroups by enumerating subsemigroups of its maximal subsemigroups.
\begin{fact}
$\Sub(S)=\big( \bigcup_{T\in \Max(S)}\Sub(T)\big)\cup \{S\}$
\end{fact}
\proof
It follows from the fact that $\Sub(S)$ is an algebraic lattice.
\qed

\begin{lemma}
Let $I$ be an ideal of $S$, then $$\Sub(S)=\big\{\langle T,U \rangle\mid T\in \Sub(S/I), U\in\Sub(I)\big\}.$$
\end{lemma}
\proof

\qed

\begin{lemma}
$T\in\Sub(S)$ and $g\in \Aut(S)$ then $g^{-1}Tg\in\Sub(S)$.
\end{lemma}
\proof
Let $s,t\in T$ and $T'=g^{-1}Tg$.
$$g^{-1}sgg^{-1}tg=g^{-1}stg.$$
\qed
\end{document}

$\cS_n$ denotes the symmetric group, $\cT_n$ the full transformation group on $n$ points.